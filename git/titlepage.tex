% !TEX program = xelatex
% 这是中国海洋大学本科毕业论文的封面模板
% 请在主文档中使用 % !TEX program = xelatex
% 这是中国海洋大学本科毕业论文的封面模板
% 请在主文档中使用 % !TEX program = xelatex
% 这是中国海洋大学本科毕业论文的封面模板
% 请在主文档中使用 % !TEX program = xelatex
% 这是中国海洋大学本科毕业论文的封面模板
% 请在主文档中使用 \input{titlepage.tex} 调用此文件

\thispagestyle{empty} % 封面不显示页码


% --- 自定义命令 ---
\newcommand{\degreeclass}{本科} % 学位类型
\newcommand{\universityname}{中国海洋大学}

% --- 封面内容 ---
\begin{titlepage}
    \centering

    % 页眉信息
    \vspace*{-2cm} % 调整与页面顶部的距离
    \vspace{1.5cm} % 与Logo之间的垂直间距

    % Logo
    \begin{flushright}
        \includegraphics[width=0.15\textwidth]{pic/ouc_logo.eps} % 请确保 OUC_Logo.png 文件存在且路径正确
    \end{flushright}
    \vspace{2cm} % 与正标题之间的垂直间距

    % 主标题
    {\zihao{2} \heiti \textbf{系统开发工具基础实验报告}}
    \par \vspace{1.5cm} % 与副标题之间的垂直间距

    % 副标题 (题目)
    {\zihao{2} \heiti \textbf{题目:}\quad \underline{版本控制(git)}}
    \par \vspace{3cm} % 与学生信息之间的垂直间距

    % 学生及指导教师信息
    \begin{tabular}{r l}
        \zihao{-4} \songti 学生姓名\quad & \underline{周洋迅} \quad 学号\quad \underline{24020007175} \\
        \zihao{-4} \songti 学部、学院(中心)\quad & \underline{信息科学与工程学部} \\
        \zihao{-4} \songti 专业\quad & \underline{计算机科学与技术} \\
        \zihao{-4} \songti 日期\quad & \underline{2025} 年 \underline{8} 月 \underline{29} 日 \\
        \zihao{-4} \songti github链接\quad & \underline{https://github.com/zysgusg/SysDevelopmentTools}\\
    \end{tabular}
    \par \vspace{2cm} % 与学校名称之间的垂直间距

    % 学校名称
    {\zihao{3} \heiti \textbf{\universityname}}
    \par \vspace{1cm} % 页面底部留白

\end{titlepage} 调用此文件

\thispagestyle{empty} % 封面不显示页码


% --- 自定义命令 ---
\newcommand{\degreeclass}{本科} % 学位类型
\newcommand{\universityname}{中国海洋大学}

% --- 封面内容 ---
\begin{titlepage}
    \centering

    % 页眉信息
    \vspace*{-2cm} % 调整与页面顶部的距离
    \vspace{1.5cm} % 与Logo之间的垂直间距

    % Logo
    \begin{flushright}
        \includegraphics[width=0.15\textwidth]{pic/ouc_logo.eps} % 请确保 OUC_Logo.png 文件存在且路径正确
    \end{flushright}
    \vspace{2cm} % 与正标题之间的垂直间距

    % 主标题
    {\zihao{2} \heiti \textbf{系统开发工具基础实验报告}}
    \par \vspace{1.5cm} % 与副标题之间的垂直间距

    % 副标题 (题目)
    {\zihao{2} \heiti \textbf{题目:}\quad \underline{版本控制(git)}}
    \par \vspace{3cm} % 与学生信息之间的垂直间距

    % 学生及指导教师信息
    \begin{tabular}{r l}
        \zihao{-4} \songti 学生姓名\quad & \underline{周洋迅} \quad 学号\quad \underline{24020007175} \\
        \zihao{-4} \songti 学部、学院(中心)\quad & \underline{信息科学与工程学部} \\
        \zihao{-4} \songti 专业\quad & \underline{计算机科学与技术} \\
        \zihao{-4} \songti 日期\quad & \underline{2025} 年 \underline{8} 月 \underline{29} 日 \\
        \zihao{-4} \songti github链接\quad & \underline{https://github.com/zysgusg/SysDevelopmentTools}\\
    \end{tabular}
    \par \vspace{2cm} % 与学校名称之间的垂直间距

    % 学校名称
    {\zihao{3} \heiti \textbf{\universityname}}
    \par \vspace{1cm} % 页面底部留白

\end{titlepage} 调用此文件

\thispagestyle{empty} % 封面不显示页码


% --- 自定义命令 ---
\newcommand{\degreeclass}{本科} % 学位类型
\newcommand{\universityname}{中国海洋大学}

% --- 封面内容 ---
\begin{titlepage}
    \centering

    % 页眉信息
    \vspace*{-2cm} % 调整与页面顶部的距离
    \vspace{1.5cm} % 与Logo之间的垂直间距

    % Logo
    \begin{flushright}
        \includegraphics[width=0.15\textwidth]{pic/ouc_logo.eps} % 请确保 OUC_Logo.png 文件存在且路径正确
    \end{flushright}
    \vspace{2cm} % 与正标题之间的垂直间距

    % 主标题
    {\zihao{2} \heiti \textbf{系统开发工具基础实验报告}}
    \par \vspace{1.5cm} % 与副标题之间的垂直间距

    % 副标题 (题目)
    {\zihao{2} \heiti \textbf{题目:}\quad \underline{版本控制(git)}}
    \par \vspace{3cm} % 与学生信息之间的垂直间距

    % 学生及指导教师信息
    \begin{tabular}{r l}
        \zihao{-4} \songti 学生姓名\quad & \underline{周洋迅} \quad 学号\quad \underline{24020007175} \\
        \zihao{-4} \songti 学部、学院(中心)\quad & \underline{信息科学与工程学部} \\
        \zihao{-4} \songti 专业\quad & \underline{计算机科学与技术} \\
        \zihao{-4} \songti 日期\quad & \underline{2025} 年 \underline{8} 月 \underline{29} 日 \\
        \zihao{-4} \songti github链接\quad & \underline{https://github.com/zysgusg/SysDevelopmentTools}\\
    \end{tabular}
    \par \vspace{2cm} % 与学校名称之间的垂直间距

    % 学校名称
    {\zihao{3} \heiti \textbf{\universityname}}
    \par \vspace{1cm} % 页面底部留白

\end{titlepage} 调用此文件

\thispagestyle{empty} % 封面不显示页码


% --- 自定义命令 ---
\newcommand{\degreeclass}{本科} % 学位类型
\newcommand{\universityname}{中国海洋大学}

% --- 封面内容 ---
\begin{titlepage}
    \centering

    % 页眉信息
    \vspace*{-2cm} % 调整与页面顶部的距离
    \vspace{1.5cm} % 与Logo之间的垂直间距

    % Logo
    \begin{flushright}
        \includegraphics[width=0.15\textwidth]{pic/ouc_logo.eps} % 请确保 OUC_Logo.png 文件存在且路径正确
    \end{flushright}
    \vspace{2cm} % 与正标题之间的垂直间距

    % 主标题
    {\zihao{2} \heiti \textbf{系统开发工具基础实验报告}}
    \par \vspace{1.5cm} % 与副标题之间的垂直间距

    % 副标题 (题目)
    {\zihao{2} \heiti \textbf{题目:}\quad \underline{版本控制(git)}}
    \par \vspace{3cm} % 与学生信息之间的垂直间距

    % 学生及指导教师信息
    \begin{tabular}{r l}
        \zihao{-4} \songti 学生姓名\quad & \underline{周洋迅} \quad 学号\quad \underline{24020007175} \\
        \zihao{-4} \songti 学部、学院(中心)\quad & \underline{信息科学与工程学部} \\
        \zihao{-4} \songti 专业\quad & \underline{计算机科学与技术} \\
        \zihao{-4} \songti 日期\quad & \underline{2025} 年 \underline{8} 月 \underline{29} 日 \\
        \zihao{-4} \songti github链接\quad & \underline{https://github.com/zysgusg/SysDevelopmentTools}\\
    \end{tabular}
    \par \vspace{2cm} % 与学校名称之间的垂直间距

    % 学校名称
    {\zihao{3} \heiti \textbf{\universityname}}
    \par \vspace{1cm} % 页面底部留白

\end{titlepage}